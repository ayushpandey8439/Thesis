% !TEX root = ../main.tex
%
\pdfbookmark[0]{Résumé}{Résumé}
\addchap*{Résumé}
\label{chap:résumé}

\vspace*{8mm}
Les hiérarchies servent de structures fondamentales dans diverses disciplines, modélisant les relations hiérarchiques en informatique, en biologie, dans les réseaux sociaux et en logistique. Cependant, les mises à jour dynamiques et simultanées dans les systèmes du monde réel nécessitent des techniques de synchronisation pour maintenir la cohérence des données malgré l'accès simultané. Cet article explore une nouvelle approche appelée CALock pour synchroniser les opérations sur les hiérarchies en utilisant un schéma d'étiquetage qui facilite le verrouillage multi-granularité.

Notre approche concerne à la fois les lectures et écritures concurrentes de données et les modifications structurelles. CALock exploite la topologie hiérarchique par le biais d'un nouveau schéma d'étiquetage permettant d'identifier les ancêtres communs des sommets. Cela permet à un thread d'identifier un granule de verrouillage approprié pour sa demande de verrouillage. L'utilisation de granularités de verrouillage variables optimise les opérations à travers la hiérarchie tout en garantissant la cohérence et les performances.

Nous présentons une discussion détaillée de l'étiquetage CALock et de l'algorithme de verrouillage, nous prouvons ses propriétés et nous l'évaluons de manière expérimentale. Sur des hiérarchies statiques, CALock reste compétitif par rapport aux schémas d'étiquetage précédents et offre une meilleure simultanéité et un meilleur débit lorsque des modifications structurelles changent la hiérarchie. En particulier, CALock améliore le débit de 4,5$\times$ et le temps de réponse de 1,5$\times$ pour les charges de travail qui contiennent des modifications structurelles.

\textbf{Mots-clés:} 
Verrouillage multi-granularité, Données hiérarchiques, Graphes, Verrouillage, Synchronisation, Topologie des graphes, Ancêtres. 