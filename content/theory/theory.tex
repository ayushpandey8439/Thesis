\chapter{Theoretical Framework} \label{chap:theory}

In this chapter, we present the formal definitions and theoretical underpinnings of CALock.
We begin by introducing the basic concepts of graphs, paths, and vertex relationships.
Next we discuss the bounds of commonality between vertices in a graph and how they can be used to determine the optimal locking grain.
Finally, we formulate the optimal locking grain problem mathematically and prove its correctness.

\section{Graphs, Paths and Vertex Relationships}


Let $G=(V, E)$ be a directed graph. $V$ is its set of vertices, connected by directed edges in the set $E \subseteq V \times V$.  A rooted graph has a distinguished vertex $r \in V$ such that all other vertices are reachable from $r$.

A pair of vertices $(u, v)$ can be connected by a sequence of edges, called the path between $u$ and $v$, noted $p$. The set of vertices of $p$ is noted $\mathcal{V}(p)$. The length $l_p$ of this path is the size of $\mathcal{V}(p)$ i.e. $l_p = \lvert \mathcal{V}(p)\rvert$.



%\textcolor{red}{For two paths $p_1$ and $p_2$, if $l_{p_1} \geq l_{p_2}$, then we say $p_1 \geq p_2$ and vice a versa.}

In the general case, several such paths may exist between a pair of vertices.
The set of paths between $u$ and $v$ is noted $\mathcal{P}_{(u,v)}$.
The depth $\delta(v)$ of a vertex $v$ is the length of the shortest path in the set $\mathcal{P}_{(r,v)}$.
% \subsection{Ancestors and Descendants}
\begin{definition}[Ancestor and Descendent]
	An ancestor of a vertex $u$ is a vertex $v$ in $G$ that lies on a path from $r$ to $u.$ The vertex $u$ is then called the descendant of vertex $v$.
	%The set of ancestors is given by the set relation $A(u)$.
	%\begin{equation*}
%		A(u)=\{v\in V|\exists p\in \mathcal{P}_{(r,u)}, v \in p\}
	%\end{equation*}
\end{definition}
% \subsection{Guarding Ancestors}
% We define the guarding ancestor(s) of a vertex.

\begin{definition}[Guarding Ancestor]
	A guarding ancestor of a vertex $u$ is a vertex $v \in G$ that lies on all paths from $r$ to $u$.
	The set of guarding ancestors of $u$ is noted $L_u$.
	%The set of single ancestors is given by the set relation $SA(u)$.
	%\begin{equation*}
	%	SA(u) =	\{v\in A(u)|\forall p\in \mathcal{P}_{(r,u)}\Rightarrow v\in p\}
	%\end{equation*}
\end{definition}

\begin{definition}[Lowest guarding Ancestor]
	The lowest guarding ancestor $LGA(u)$ of a vertex $u$ is a guarding ancestor of $u$ with the maximum depth.
\end{definition}

%\begin{definition}[LGA-Tree]
%	The LGA-tree $T_G$ of $G$  is an auxiliary tree that has the same vertices $V$, and whore edges are defined such that the parent vertex of $v \neq r$ is the LGA of $v$ in $G$.
%\end{definition}


% \subsection{Common Ancestors} \label{cadefinitions}
% We define the common ancestor(s) of a set of vertices.

% This can be done trivially for trees since paths between the root of the tree and the target vertex are unique \cite{aho1973finding}.

\begin{definition}[Common Ancestor] \label{def:commonAncestor}
	A common ancestor (CA) of two vertices $u$ and $v$ is a vertex $c$ that is an ancestor of both $u$ and $v$.
	%The set of common ancestors of
%	\begin{equation*}
	%	CA(u) =	\{c \in V | c \in A(u) , c \in A(v)\}
	%\end{equation*}
\end{definition}

\begin{definition}[Lowest Common Ancestor] \label{def:LowestCommonAncestor}
	The lowest common ancestor $LCA(u,v)$ of two vertices $u$ and $v$ is a common ancestor of $u$ and $v$ with the maximum depth.
\end{definition}

% \subsection{Lowest Guarding Common Ancestor} \label{LSCADefinitions}


The Lowest Guarding Common Ancestor $LGCA(Q)$ of a set of vertices $Q$ is the deepest vertex that is both a guarding ancestor and a common ancestor of all vertices in $Q$. Fischer et. al \cite{fischer2010new} derive the following relationships between the LGA and the LGCA of vertices in a rooted DAG.


%\begin{lemma}\label{lscaislca}
%	Let $G$ be a DAG, rooted at r, and $T_G$  its corresponding LGA-tree. Further, let v, w $\in$ V be two arbitrary vertices in G. Then $LGCA_G(v,w) = LCA_{T_G}(v,w)$
%\end{lemma}

\begin{lemma}\label{lgaislgcaofparents}
	\begin{align*}
		\text{For a vertex } v \in G: v \neq r , LGA_G(v) = LGCA_G(\mathit{parents}_G(v))
	\end{align*}

\end{lemma}


\begin{definition}\label{associativelsca}
	\begin{align*}
		LGCA_G (w_1, ... ,w_k) = LGCA_G (w_1, LGCA_G (w_2, ... ,w_k))
	\end{align*}

\end{definition}

%\begin{definition}\label{associativelca}
%\begin{align*}
%	LCA_{T_G} (u_1, u_2,..., u_n) = LCA_{T_G} (u_1, LCA_{T_G} (u_2,...,u_n))
%\end{align*}
%\end{definition}

% \subsection{\ayush{Characterstic sets} Sets of Single ancestors}
% We use the LGA-tree $T_G$ to label the vertices of a rooted DAG \emph{G}.
% A vertex $u$ is labelled with an ordered set containing the vertices on the path from the root of $T_G$ to $u$.
% Since $T_G$ is a tree, this path is unique and so is the label $L_u$.

% In order to compute the label of a vertex $u$ in $G$ without needing to compute the LGA-tree $T_G$, we take the set intersection of the labels of $u$'s ancestors in $G$.
The $LGCA$ of a set of vertices is the intersection of the sets of guarding ancestors of those vertices. We have the following theorem:

\begin{theorem} \label{proofOfDeepness}
	\begin{gather*}
		LGCA_G(v,w) \text{ is the vertex of the maximum depth in } L_v\cap L_w
	\end{gather*}
\end{theorem}

\begin{proof}
	 Let $l = LGCA_G(v,w)$ be the LGCA of $\{v, w\}$, We need to show that $l$ is the deepest vertex in  $L_v \cap L_w$.
%	 Lowest common ancestor of  in the LGA tree $T_G$. By lemma \ref{lscaislca}, we can say that $l = LGCA_G(v,w)$.


	Let's assume that there is a vertex $l' \neq l$ that lies on all paths from $l$ to $v$ and $l$ to $w$ in $G$.
	Since $l'$ lies on the paths in the sets $\mathcal{P}_{(l,v)}$ and $\mathcal{P}_{(l,w)}$, inductively, $l'$ also lies on all the paths from the root ($r$) of the graph to the vertices $v$ and $w$.
	If  $l'$ lies on these paths then it is a guarding ancestor of $v$ and $w$ and the following holds true: $(l' \in L_v) \land (l' \in L_w) \equiv l' \in L_v \cap L_w$.

	Since $l'$ lies on the paths to $v$ and $w$ after $l$, by the definition of depth $\delta(v)$ of a vertex,	$\delta(l') > \delta(l)$.

	Now we have two conclusions, $l' \in L_v \cap L_w$ and $\delta(l') > \delta(l)$.

	Since $l'$ is deeper than $l$, it should be the deepest member of  $ L_v \cap L_w$.
	Which means, $l'$ is the LGCA of $v$ and $w$.
	But this contradicts our original assumption that $l$ is the LGCA of $v$ and $w$.
	This means our assumptions on $l'$ contradict the definition $l = LGCA_G(v,w)$ and $l$ is the deepest element in the set $L_v \cap L_w$.
\end{proof}

\begin{table}[ht]
    \centering
	\captionsetup{justification=centering}
	\caption{Terms used in the definitions}
	\begin{tabular}{c l}
		\textbf{Term} & \textbf{Meaning} \\
		\hline 
		$G$ & Graph \\
		$V$ & Vertices of $G$\\
		$E$ & Edges of $G$\\
		$r$ & Root of $G$\\
		$u, v, w$& Vertices \\
		$\delta(v)$ & Depth of vertex $v$\\
		$GA(u)$ & Guarding ancestors of $u$\\
		$LGA(u)$ & Lowest guarding ancestor of $u$\\
		$Q$ & Set of vertices\\
		$CA(Q)$ & Common ancestors of $Q$\\
		$LCA(u,v)$ & Lowest common ancestor of $u$ and $v$\\
		$LGCA(Q)$ & Lowest guarding common ancestor of $Q$\\
		$L_u$ & Label of vertex $u$\\
	\end{tabular}\\
	\label{definitionsLegend}
\end{table}

\section{CALock labelling scheme} \label{sec:labellingScheme}
CALock label for a vertex is the set of its guarding ancestors. 
To ease the implementation, we derive a recursive function from the definitions in Section \ref{DefinitionsandProof} that can be used to compute these labels. 
An implementation of this function is shown in Algorithm \ref{labelAssignment}
\subsection{Recursive labelling function} \label{labellingScheme}
Theorem \ref{proofOfDeepness} is defined for rooted DAGs. We combine it with the definitions and lemmas from Section \ref{LSCADefinitions} to derive a recursive function that we use to label the vertices in a graph. To this end,
Lemma \ref{lgaislgcaofparents} can be rewritten using Definition \ref{associativelsca} as follows:
\begin{equation}\label{eqn1}
\begin{split}
    &{LGA_G(v) = LGCA_G (p_1, LGCA_G (p_2,...,p_k))}
    \\ &\text{where } p_1, p_2, ... p_k \text{ are the parents of } v
\end{split}
\end{equation}

%Definition \ref{associativelsca} can be rewritten using Theorem \ref{proofOfDeepness} as follows:
%\begin{equation}\label{eqn2}
%\begin{split}
%    &LSA_G (u_1, .... , u_n)  = LGCA_G(u_1, ... , u_n)
%    \\ &\text{which is the deepest element in } L_{u_1} \cap ... \cap L_{u_n}
%\end{split}
%\end{equation}

On combining equations \ref{eqn1} and theorem \ref{proofOfDeepness}; $LGA_G(v)$ is the deepest vertex in $ L_{p_1} \cap L_{p_2} \cap ... \cap L_{p_k}$ where $p_1, p_2, ... p_k$ are the parents of $v$.

Therefore, the sets of guarding ancestors of parents of $v$, are enough to compute the lowest guarding ancestor of $v$. 
We use this property to recursively compute the labels of the vertices in a graph using the following function.

\begin{equation}\label{labellingRecursion}
	L_v = 
	\begin{cases}
		\{v\} & \text{v is the root} \\
		\{v\} \cup \{ \cap_{u\in parents(v)} L_u\} & \text{otherwise}
	\end{cases}
\end{equation}


\subsection{Labelling graphs recursively}

\subsubsection{Labelling acyclic graphs}
Labelling starts at the root of the graph when the function \inline|AssignLabel(r)| is called. It uses Relation \ref{labellingRecursion} implemented in lines \ref{labellingRelationImplBegin} - \ref{labellingRelationImplEnd} to assign the labels. The root vertex does not have parents, so the label of the root is the set containing the ID of the root itself (lines \ref{rootLabellingBegin} - \ref{rootLabellingEnd}). For example, in figure \ref{calockexample} vertex $A$ has the label $\{A\}$.

Then, the children of the vertex v are explored via a breadth-first traversal over the graph. Using the labels of the parents, the label of the child is computed via Relation \ref{labellingRecursion}. For example, in figure \ref{calockexample}, the label of vertex $B$ is $\{A, B\}$ since it has only one parent. Vertex $E$ has two parents which have the labels $\{A, B\}$ and $\{A, C\}$ respectively. Their set intersection is $\{A\}$. Using Relation \ref{labellingRecursion}, the label of $E$ is $\{A, E\}$.The graph is explored and labelled until the recursion reaches a fix-point and terminates (line \ref{recursionCheck}). 

\subsubsection{Labelling strongly connected components}

Theorem \ref{proofOfDeepness} assumes that the graph does not contain strongly connected components. In real-life applications, maintaining this constraint is difficult. There are strategies to find strongly connected components in a directed graph \cite{sharir1981strong, tarjan1972depth, cheriyan1996algorithms} that can be contracted by vertex contraction \cite{walsh2006hub}. We, however, maintain that finding and eliminating strongly connected components is not required to identify lock grains using CALock when the labels are assigned using Relation \ref{labellingRecursion}. 
%This is because the label of a vertex in the connected component reaches a fix-point based on the topology of the graph.

% This is because when labelling with recursive Relation \ref{labellingRecursion}, the label of a vertex in the connected component reaches a fix-point based on the topology of the graph.

Assume that $N \subseteq V$ is a set of vertices of a graph that are strongly connected. Since every vertex in $N$ is reachable from every other vertex in $N$, the path set $\mathcal{P}_{(u,v)}$ for any pair of vertices $u, v\in N$ contains the same vertices in different orders and hence, every vertex has the same set of guarding ancestors. The function \inline|BFLabel| recurses over vertices in $N$ until all the paths to a vertex are explored and its label does not change anymore (line \ref{recursionCheck}). When \inline|BFLabel| terminates, all vertices in $N$ have the same label.

Therefore, Relation \ref{labellingRecursion} is sufficient to label rooted directed graphs with strongly connected components. An implementation of recursive labelling is shown in Algorithm \ref{labelAssignment}.

\subsubsection{Relabelling graphs} The topology of a graph can change due to \emph{structural modifications}. 
Then, the labels of the vertices need to be recomputed. 
The same function \inline|AssignLabel(v)| can be used to relabel the graph. 
Relabelling begins at the vertex affected by the structural modification via the function \inline|AssignLabel(v)|. 
In the same fashion as the initial labelling, the parents of the affected vertex are used to compute its label and then recursively of its children. Relabelling is terminated as soon as it reaches a fixpoint (line \ref{recursionCheck}).  





\begin{algorithm}
	\caption{Labelling the graph}
	\begin{algorithmic}[1]
		\Procedure{AssignLabels}{v}
		\State $queue$.\Call{push}{v}
		\While{$queue$.\Call{hasNext}{ }}
		\State $v\gets queue.$\Call{Next}{ }
		\State \Call{BFLabelA}{v, queue}
		\EndWhile
		\EndProcedure
		\Statex

		\Procedure{BFLabelA}{v, queue}
		\State $C\gets$\Call{children}{v}
		\If{$parents(v) = \emptyset$ } \label{rootLabellingBegin}
		\State{$v.label\gets\{v\}$}
		\State $queue.$\Call{push}{ $C$ }
		\State return
		\EndIf\label{rootLabellingEnd}
		\State $P\gets$\Call{parents}{v}\label{labellingRelationImplBegin}
		\State $tempLabel\gets$\Call{Intersection}{P.$labels$}\label{intersectionStart}
		\State $tempLabel$.\Call{append}{v} \label{intersectionEnd}
		\If{$v.label \neq tempLabel$} \label{recursionCheck}
		\State $v.label\gets tempLabel$
		\State $queue.$\Call{push}{ $C$ }
		\EndIf \label{labellingRelationImplEnd}
		\EndProcedure
	\end{algorithmic}
	\label{labelAssignment}
\end{algorithm}
