% !TEX root = ../main.tex
%


\chapter{Introduction}
\adjustmtc


In database systems, maintaining data consistency and integrity during concurrent access is a foundational challenge. As systems grow in scale and complexity, managing data concurrency effectively becomes critical, especially in scenarios where multiple transactions access shared data simultaneously. Locking mechanisms provide a structured approach to concurrency control, with multi-granularity locking (MGL) being a particularly valuable technique. MGL has been widely studied in traditional database settings, especially for table spaces organized in tree-structured data, where it efficiently manages hierarchical access patterns.

However, as data models evolve to incorporate intricate hierarchies and highly connected structures like graphs and networks, traditional approaches to MGL require adaptation. Hierarchical data models are common in domains such as file systems, organizational structures, ontologies, and XML databases. These models naturally capture relationships across different levels of abstraction, where data elements can be accessed concurrently by multiple users or processes. Such concurrent access raises the possibility of conflicts and inconsistencies, particularly when different parts of the hierarchy or network are accessed simultaneously.

To address these issues, MGL provides a mechanism to manage locks at varying levels of granularity, allowing coarse-grained locks at higher levels of the hierarchy and fine-grained locks at lower levels. This flexibility supports efficient concurrency control by enabling transactions to lock large portions of data when broad access is required or smaller segments for more specific operations. However, implementing MGL effectively in systems with complex hierarchies and interconnected data structures introduces unique challenges. Balancing performance with fairness becomes crucial, as the system’s performance depends on minimizing lock contention and avoiding deadlocks, while ensuring that locks at different levels can coexist without causing unnecessary blocks or resource wastage.

In summary, as data structures grow more intricate, a deeper understanding of multi-granularity locking tailored to hierarchical and graph-based data is essential for achieving scalable, efficient concurrency control. This research aims to address these complexities, exploring adaptations of MGL that can support parallelism and data integrity in environments characterized by complex, multi-level connections.


% \section{Background and Motivation}
% Overview of concurrency control and the importance of locking in database systems. Challenges associated with multi-granularity locking. Introduction to CALock and its significance.

% \section{Research Objectives}
% Define the main goals of the research. Specific objectives related to CALock and optimal locking grain.

\section{Problem Statement}
While multi-granularity locking (MGL) is widely applied in simple hierarchical data models, adapting it to handle complex, irregular, and deeply nested hierarchies remains challenging. Traditional MGL approaches often face scalability issues in environments where hierarchies expand significantly in size and depth, or where relationships become less rigid, as in graph-like data models. One core challenge lies in efficiently determining the optimal placement of MGL locks within these intricate structures. This process must consider both the structural depth and irregularity of the hierarchy, which can make lock placement a computationally intensive task.

Furthermore, once locks are set, identifying and managing conflicts between them becomes increasingly complex, especially as parallel transactions interact at multiple levels of the hierarchy. These interactions can introduce significant synchronization challenges, particularly in distributed environments with high latency or frequent partitioning. Addressing these issues requires an extension of existing MGL frameworks that supports not only efficient lock placement and conflict detection but also maintains high performance and data consistency across complex hierarchical structures.


\section{Contributions}

In this work, we present CALock, a novel MGL protocol designed to address the challenges of concurrency control in complex hierarchical data models. CALock extends the traditional MGL framework to support hierarchies with arbitrary depth and complexity, enabling efficient and scalable concurrency management.

MGL techniques establish the granularity at which a transaction acquires a lock, to optimize access based on the transaction's needs. In CALock, we use a topological partial ordering of the vertices in the hierarchy to establish this granularity. This ordering is defined through a vertex labelling scheme. At runtime, these labels help determine the most suitable lock granularity for a given lock request based on the transaction’s access pattern.

CALock attempts to minimize the granularity of a lock, ensuring that each lock covers only the necessary portion of a hierarchy. This approach avoids fixed vertex ordering and the need for frequent relabelling, while still providing the same locking guarantees as other MGL techniques. Each vertex label consists of its set of common ancestors, computed recursively through breadth-first traversal, allowing CALock to achieve flexible and efficient locking in complex, dynamic hierarchies.


% Existing MGL techniques like intention locks \cite{gray1975granularity} use the topology of the hierarchy to detect lock conflicts but require multiple traversals from the root of the graph to the vertices being locked. 
% This makes them expensive for non-tree-like hierarchies like DAGs. Label based techniques \cite{kalikar2016domlock,anjuMID,kalikar_toggle_2019,FlexiGran2024} use an ordering of vertices to assign labels that eliminate the need for traversals. However these have poor performance over dynamic graphs due to the overhead of relabelling. 


% In this work, we present CALock, a novel multi-granularity locking protocol designed to address the challenges of managing concurrency in complex hierarchical data models.
% CALock extends the traditional multi-granularity locking framework to support hierarchies with arbitrary levels of depth and complexity, enabling efficient and scalable concurrency control.
% We use a topological partial ordering of the vertices of a hierarchy to define the lock levels and grains, allowing transactions to acquire locks at different levels of granularity based on their access patterns. 
% This topological ordering is determined via a labelling scheme that assigns unique labels to each vertex in the hierarchy which are used at runtime to determine the appropriate lock grain for a lock request. 



% CALock's labelling efficiently computes the closest common ancestor of a set of vertices in a rooted directed graph. 
% By choosing the closest common ancestor of a set of lock targets as their guard, CALock minimizes grain size. 
% In using paths, by avoiding a fixed ordering of vertices, CALock circumvents expensive relabelling while providing the same locking guarantees as other MGL techniques. 
% The label of a vertex in the graph is a set of its common ancestors, computed recursively via breadth-first traversal.



\section{Thesis Structure}
The remainder of this thesis is organized as follows:

 Chapter \ref{chap:background} presents an overview of concurrency control, multi-granularity locking and the challenges associated with complex hierarchical data models. 

 Chapter \ref{chap:relatedwork} reviews existing research in the field of multi-granularity locking, highlighting the limitations of current techniques. 

 Chapter \ref{chap:theory} provides a theoretical framework for CALock which includes the concept of lowest common ancestors in graphs, and the problem formulation for optimal locking grain selection and the proof of optimality of CALock labelling.

 Chapter \ref{chap:calock} introduces CALock, our novel multi-granularity locking protocol, and describes its design and implementation. We also discuss the properties of CALock like safety, liveness and fairness.


 
\todo{Fix this and add missing chapters}

 Chapter \ref{chap:evaluation} presents an experimental evaluation of CALock, comparing its performance against state-of-the-art techniques in a variety of scenarios using both, micro-benchmarks and macro-benchmarks.
 
 Finally, Chapter \ref{chap:conclusion} concludes the thesis, summarizing our contributions and outlining future research directions in the field of multi-granularity locking for complex hierarchical data models.

% \begin{itemize}
%     \item Chapter 2 provides an overview of the background and motivation for this work, including a discussion of the challenges associated with multi-granularity locking in complex hierarchical data models.
%     \item Chapter 3 presents a detailed review of related work in the field of multi-granularity locking, focusing on existing techniques and their limitations in managing concurrency in hierarchical data models.
%     \item Chapter 4 introduces CALock, our novel multi-granularity locking protocol designed to address the challenges of concurrency control in complex hierarchical data models. We describe the design and implementation of CALock, and evaluate its performance in comparison to existing techniques.
%     \item Chapter 5 presents an experimental evaluation of CALock, comparing its performance against state-of-the-art multi-granularity locking techniques in a variety of scenarios. We analyze the results and discuss the implications of our findings.
%     \item Chapter 6 concludes the thesis, summarizing our contributions and outlining future research directions in the field of multi-granularity locking for complex hierarchical data models.

% \chapter{Applications and Case Studies}
% \section{Application Scenarios}
% Potential real-world applications of CALock in database systems.

% \section{Case Studies}
% Detailed analysis of specific case studies where CALock was applied.

% \section{Lessons Learned}
% Insights gained from practical applications.





% \chapter{Introduction}
% \label{sec:intro}

% main intro part, in general it should contain at least 4 paragraphs each answers one of the following questions.

% \emph{P1: What's the thesis main problem?}

% \emph{P2: Why the problem is a problem?}

% \emph{P3: what's the solution?}

% \emph{P4: Why the solution is better than the state of the art?}

% \section{Overview}
% \label{sec:intro:overview}

% In the first part of this thesis,
% we explore ...

% In the second part of the thesis,
% we present \system{},
% a ...
% To answer this challenge, \system{} takes a hybrid approach, and
% ...

% To address the requirements of ..., \system{}
% enables ...

% Finally, this thesis addresses a number of design and implementation
% challenges, including ...

% \section{Contributions}
% \label{sec:intro:contributions}

% The main results of this dissertation are as follows:
% \begin{itemize}[leftmargin=*]
% \item
%   R1
% \item
%   R2
% \item
%   R3
% \item
%   R4
% \item
%   R5
% \end{itemize}

% Our experimental evaluation shows that:...

% \section{Publications}
% \label{sec:intro:publications}

% Some of the results presented in this thesis have been published as follows:

% \begin{itemize}
%   \item myFirstPublication %\cite{toumlilt:hal-03353663}
%   \item mySecondPublication %\cite{toumlilt:hal-01860334}
% \end{itemize}

% During my thesis, I explored other directions and collaborated in several projects that have helped me to get insights on the challenges of ... These efforts have led me to contribute to the following publications and deliverables:

% \begin{itemize}
%   \item TechReport %\cite{LiRAv09p46:online}
%   \item ANRProjectDelivrable %\cite{D61Newco32:online}
%   \item EUProjectDelivrable %\cite{D62Newco60:online}
% \end{itemize}

% \section{Organization}
% \label{sec:intro:organisation}

% This thesis is divided into three parts. 
% The rest of this document is organized as follows:
% \begin{itemize}
%   \item Part~\ref{part:background} introduces the common background of our work,
%   formulate the problem, presents the existing solutions and discusses the use-case
%   requirements, this part is divided into three chapters:
%   \begin{itemize}
%     \item Chapter~\ref{ch:my-domain} provides a complete and up-to-date 
%     review of the MyDomain.
%     \item Chapter~\ref{ch:requirements} presents a use-case point of view of the
%     ...
%     \item Chapter~\ref{ch:related-work} studies and compares the solutions that have
%     been designed in the state of the art of ...
%   \end{itemize}
%   \item Part~\ref{part:contributions}, in light of what we saw in the existing work, 
%   we will here justify some protocol choices used in our approach...
%   \item Part~\ref{part:evaluation} provides an experimental evaluation 
%   demonstrating the benefits of our approach, compared to other solutions 
%   from the state of the art.
%   \item the last part~\ref{part:conclusion} we summarize our contribution, and 
%   present our vision for the future requirements towards more ...
% \end{itemize}
