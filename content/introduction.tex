% !TEX root = ../main.tex
%

\chapter{Introduction}

In the realm of database systems, ensuring consistency and integrity during concurrent access to data is a critical challenge. 
As systems grow in scale and complexity, the management of data concurrency becomes increasingly significant, particularly in environments where multiple transactions access shared data simultaneously. 
A key approach to addressing this challenge is the use of locking mechanisms to control access to data, with multi-granularity locking (MGL) being one such technique.
In traditional database systems, the concept of multi-granularity locking has been well-researched for table spaces, particularly in relation to tree-structured data. 
However, as data models evolve to include complex hierarchies and highly connected structures such as graphs and networks, a more nuanced understanding of MGL is required to ensure efficient and scalable concurrency control.

Hierarchical data models are pervasive across various domains, ranging from file systems and organizational charts to ontologies and XML databases. 
Such models inherently capture relationships between entities, where the data is organized at different levels of abstraction. 
In these settings, multiple users or processes may attempt to access different parts of the hierarchy concurrently, leading to potential conflicts and the risk of data inconsistency. 
At the heart of this issue lies the challenge of efficiently managing locks across different levels of the hierarchy, while minimizing overhead and ensuring that transactions can proceed in parallel whenever possible.

Multi-granularity locking addresses this challenge by allowing locks to be set at different levels of granularity—ranging from coarse-grained locks at higher levels of the hierarchy to fine-grained locks at lower levels. 
This approach provides flexibility, enabling transactions to lock larger portions of the data when broad access is required, or smaller portions when fine-grained control is needed. 
However, implementing multi-granularity locking in systems with complex hierarchies introduces significant challenges, especially when balancing performance with correctness and fairness. 
The performance of the system is contingent on minimizing lock contention and deadlocks while ensuring that locks at different granularities can coexist without leading to unnecessary delays or resource wastage


% \section{Background and Motivation}
% Overview of concurrency control and the importance of locking in database systems. Challenges associated with multi-granularity locking. Introduction to CALock and its significance.

% \section{Research Objectives}
% Define the main goals of the research. Specific objectives related to CALock and optimal locking grain.

\section{Problem Statement}
While multi-granularity locking is widely applied in simple hierarchical data models, its adaptation to more complex, irregular and deeply nested hierarchies remains under-explored. 
Traditional approaches often fail to scale efficiently in scenarios where hierarchies grow in size and depth, or where the hierarchical relationships become less rigidly defined, such as in graph like data models. 
Additionally, the interaction between locks at different levels of the hierarchy can lead to intricate synchronization issues, particularly in highly parallel and distributed environments where latency and partitioning are significant factors.
Therefore, there is a need for new approaches that extend existing multi-granularity locking frameworks to support these complex hierarchical structures, while maintaining high levels of performance and ensuring data consistency.


\section{Contributions}
In this work, we present CALock, a novel multi-granularity locking protocol designed to address the challenges of managing concurrency in complex hierarchical data models.
CALock extends the traditional multi-granularity locking framework to support hierarchies with arbitrary levels of depth and complexity, enabling efficient and scalable concurrency control.
We use a topological partial ordering of the vertices of a hierarchy to define the lock levels and grains, allowing transactions to acquire locks at different levels of granularity based on their access patterns. 
This topological ordering is determined via a labelling scheme that assigns unique labels to each vertex in the hierarchy which are used at runtime to determine the appropriate lock grain for a lock request. 

Existing MGL techniques like intention locks \cite{gray1975granularity} use the topology of the hierarchy to detect lock conflicts but require multiple traversals from the root of the graph to the vertices being locked. 
This makes them expensive for non-tree-like hierarchies like DAGs. Label based techniques \cite{kalikar2016domlock,anjuMID,kalikar_toggle_2019,FlexiGran2024} use an ordering of vertices to assign labels that eliminate the need for traversals. However these have poor performance over dynamic graphs due to the overhead of relabelling. 


CALock's labelling efficiently computes the closest common ancestor of a set of vertices in a rooted directed graph. 
By choosing the closest common ancestor of a set of lock targets as their guard, CALock minimizes grain size. 
In using paths, by avoiding a fixed ordering of vertices, CALock circumvents expensive relabelling while providing the same locking guarantees as other MGL techniques. 
The label of a vertex in the graph is a set of its common ancestors, computed recursively via breadth-first traversal.



\section{Thesis Structure}
The remainder of this thesis is organized as follows:

 Chapter \ref{chap:background} presents an overview of concurrency control, multi-granularity locking and the challenges associated with complex hierarchical data models. 

 Chapter \ref{chap:relatedwork} reviews existing research in the field of multi-granularity locking, highlighting the limitations of current techniques. 

 Chapter \ref{chap:theory} provides a theoretical framework for CALock which includes the concept of lowest common ancestors in graphs, and the problem formulation for optimal locking grain selection and the proof of optimality of CALock labelling.

 Chapter \ref{chap:calock} introduces CALock, our novel multi-granularity locking protocol, and describes its design and implementation. We also discuss the properties of CALock like safety, liveness and fairness.

 Chapter \ref{chap:evaluation} presents an experimental evaluation of CALock, comparing its performance against state-of-the-art techniques in a variety of scenarios using both, micro-benchmarks and macro-benchmarks.
 
 Finally, Chapter \ref{chap:conclusion} concludes the thesis, summarizing our contributions and outlining future research directions in the field of multi-granularity locking for complex hierarchical data models.

% \begin{itemize}
%     \item Chapter 2 provides an overview of the background and motivation for this work, including a discussion of the challenges associated with multi-granularity locking in complex hierarchical data models.
%     \item Chapter 3 presents a detailed review of related work in the field of multi-granularity locking, focusing on existing techniques and their limitations in managing concurrency in hierarchical data models.
%     \item Chapter 4 introduces CALock, our novel multi-granularity locking protocol designed to address the challenges of concurrency control in complex hierarchical data models. We describe the design and implementation of CALock, and evaluate its performance in comparison to existing techniques.
%     \item Chapter 5 presents an experimental evaluation of CALock, comparing its performance against state-of-the-art multi-granularity locking techniques in a variety of scenarios. We analyze the results and discuss the implications of our findings.
%     \item Chapter 6 concludes the thesis, summarizing our contributions and outlining future research directions in the field of multi-granularity locking for complex hierarchical data models.

% \chapter{Applications and Case Studies}
% \section{Application Scenarios}
% Potential real-world applications of CALock in database systems.

% \section{Case Studies}
% Detailed analysis of specific case studies where CALock was applied.

% \section{Lessons Learned}
% Insights gained from practical applications.





% \chapter{Introduction}
% \label{sec:intro}

% main intro part, in general it should contain at least 4 paragraphs each answers one of the following questions.

% \emph{P1: What's the thesis main problem?}

% \emph{P2: Why the problem is a problem?}

% \emph{P3: what's the solution?}

% \emph{P4: Why the solution is better than the state of the art?}

% \section{Overview}
% \label{sec:intro:overview}

% In the first part of this thesis,
% we explore ...

% In the second part of the thesis,
% we present \system{},
% a ...
% To answer this challenge, \system{} takes a hybrid approach, and
% ...

% To address the requirements of ..., \system{}
% enables ...

% Finally, this thesis addresses a number of design and implementation
% challenges, including ...

% \section{Contributions}
% \label{sec:intro:contributions}

% The main results of this dissertation are as follows:
% \begin{itemize}[leftmargin=*]
% \item
%   R1
% \item
%   R2
% \item
%   R3
% \item
%   R4
% \item
%   R5
% \end{itemize}

% Our experimental evaluation shows that:...

% \section{Publications}
% \label{sec:intro:publications}

% Some of the results presented in this thesis have been published as follows:

% \begin{itemize}
%   \item myFirstPublication %\cite{toumlilt:hal-03353663}
%   \item mySecondPublication %\cite{toumlilt:hal-01860334}
% \end{itemize}

% During my thesis, I explored other directions and collaborated in several projects that have helped me to get insights on the challenges of ... These efforts have led me to contribute to the following publications and deliverables:

% \begin{itemize}
%   \item TechReport %\cite{LiRAv09p46:online}
%   \item ANRProjectDelivrable %\cite{D61Newco32:online}
%   \item EUProjectDelivrable %\cite{D62Newco60:online}
% \end{itemize}

% \section{Organization}
% \label{sec:intro:organisation}

% This thesis is divided into three parts. 
% The rest of this document is organized as follows:
% \begin{itemize}
%   \item Part~\ref{part:background} introduces the common background of our work,
%   formulate the problem, presents the existing solutions and discusses the use-case
%   requirements, this part is divided into three chapters:
%   \begin{itemize}
%     \item Chapter~\ref{ch:my-domain} provides a complete and up-to-date 
%     review of the MyDomain.
%     \item Chapter~\ref{ch:requirements} presents a use-case point of view of the
%     ...
%     \item Chapter~\ref{ch:related-work} studies and compares the solutions that have
%     been designed in the state of the art of ...
%   \end{itemize}
%   \item Part~\ref{part:contributions}, in light of what we saw in the existing work, 
%   we will here justify some protocol choices used in our approach...
%   \item Part~\ref{part:evaluation} provides an experimental evaluation 
%   demonstrating the benefits of our approach, compared to other solutions 
%   from the state of the art.
%   \item the last part~\ref{part:conclusion} we summarize our contribution, and 
%   present our vision for the future requirements towards more ...
% \end{itemize}
